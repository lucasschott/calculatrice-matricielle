\documentclass[a4paper,11pt,DIV=11]{scrartcl}
\usepackage[utf8]{inputenc}
\usepackage[T1]{fontenc}
\usepackage[francais]{babel}
\usepackage{setspace}
\usepackage{graphicx}
\usepackage{dirtytalk}


%opening
\title{Rapport de projet d'algèbre linéaire}
\author{Lucas Schott}

\begin{document}

\begin{onehalfspace}

    \maketitle

    \section*{Introduction}


    Le but de ce projet est de réaliser une mini-bibliothèque de calculs sur les matrices en langage C,
    ainsi qu'un programme permettant de créer des matrices et d'utiliser toutes les fonctions de 
    la bibliothèque sur ces matrices. Ce projet a été réalisé seul.\\
    Les fonctions réalisées sont:
    \begin{itemize}
        \item le calcul de la transposée d'une matrice.
        \item la somme de deux matrices.
        \item la multiplications de deux matrices.
        \item le calcul du déterminant d'une matrice carré:
            \begin{itemize}
                \item[-] en utilisant un algorithme récursif.
                \item[-] par une mèthode optimisée si la matrice est triangulaire.
            \end{itemize}
        \item le calcul de l'inverse d'une matrice:
            \begin{itemize}
                \item[-] par la comatrice.
                \item[-] par le pivot de Gauss.
            \end{itemize}
        \item la résolution d'un système d'équations linéaire par pivot de Gauss.
        \item la décomposition PLU.\@
        \item le calcul des valeurs propres d'une matrice carré 2*2.
    \end{itemize}


    \section*{Utilisation}


    Pour compiler le code source il suffit de lancer \emph{make} dans le répertoire racine du projet,
    l'éxecutable se trouve dans le répertoire \emph{bin/} du répertoire du projet.
    
    Pour utiliser le programme, une fois le programme lancé il suffit de suivre les instructions affichés.
    Un menu sera affiché avec les différentes opérations réalisables, il suffira de marquer le numéro de 
    l'opération à effectuer puis d'appuier sur \emph{entrée}. Le programme demandera ensuite d'entrer
    la/les matrice(s) necessaire(s) pour effectuer l'opération sélectionnée.

    Pour créer une matrice, le programme demande en premier le nombre de lignes de la matrice, puis le
    nombre de colonnes. Une fois ces informations données, il demandera la valeur de chaque élément de la
    matrice un par un, en commençant en haut à gauche de la matrice et en allant vers la droite, puis
    vers le bas.

    Si une saisie est incorrecte le programme renvera l'utilisateur au menu principal.


    \section*{Implémentation}


    Le programme a été réalisé en langage C, il est composé de trois fichiers \say{.c} et de deux fichiers
    \say{.h}: matrice.c et matrice.h définissent les structures de données et les fonctions sur les matrices,
    io.c et io.h contiennent les fonctions d'entrée et de sortie (menus, affichage des matrices, \ldots), et
    main.c contient l'interface utilisateur.\\ 

    \subsection*{Structure de donnée}

    une structure \emph{Matrix} qui contient:
    \begin{itemize}
        \item un tableau de double \emph{mat} qui contient les elements de la matrice
        \item une entier \emph{nb\_rows} qui représente le nombre de lignes de la matrice.
        \item un entier \emph{nb\_columns} qui représente le nombre de colonnes de la matrice.
        \item un booléen \emph{valide} qui vaut \say{true} si la matrice est valide, il peut valoir
            \say{false} si la matrice est le résultat d'un calcul qui n'a pas pu aboutir, comme
            l'inverse d'une matrice de déterminant nul par exemple.
    \end{itemize}



    \section*{Conclusion}

    Ce projet a été très long mais aussi très interressant à réaliser. Les premières fonctions ont été
    faites assez rapidement, mais certaines comme l'inversion de la matrice par pivot de Gauss ou la résolution
    de systèmes linéaires m'ont pris pas mal de temps. L'interface utilisateur a aussi été très longue à
    réaliser. J'ai encore voulu, à la fin,  implémenter un système de variables pour sauvegarder des matrices
    et pouvoir les utiliser dans n'importe quelle fonction sans devoir la re-saisire, mais par manque de temps
    de je n'ai pu implementer celà. J'ai effectué de nombreux testes sur mon programme, et à part quelques
    fuites mémoire que je n'ai pas eu le temps de résoudre, il est fonctionnel.


\end{onehalfspace}

\end{document}
